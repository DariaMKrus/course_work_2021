\begin{equation}\label{eq:1.6}
    \begin{gathered}
        \sum_{i=1}^n v_i\frac{\partial v}{\partial x_j}-\nu\Delta v- 2\varkappa Div\bigg(v_k\frac{\partial\mathcal{E}(v)}
        {\partial x_k}\bigg)-\\-2\varkappa Div\bigg(\mathcal{E}(v)W_p(v)-W_p(v)\mathcal{E}(v)\bigg)+grad p=f, x\in\Omega
    \end{gathered}
\end{equation}

\begin{equation}\label{eq:1.7}
    \begin{gathered}
        div \ v=0, x\in\Omega
    \end{gathered}
\end{equation}

Для системы (\ref{eq:1.6})"=(\ref{eq:1.7}) рассмотрим краетвую задачу с граничным условием
\begin{equation}\label{eq:1.8}
     \begin{gathered}
         v\mid_{\partial\Omega}=0.
     \end{gathered}
\end{equation}

В настоящей работе исследуется существование слабого решения краевой задачи (\ref{eq:1.6})"=(\ref{eq:1.8}), описывающего движение слабых
водных растворов полимеров, заполняющих ограниченную область $\Omega \in R^n$, $n = 2,3$, которое определяется реологическим
соотношением  со сглаженной производной Яуманна.

Для исследования используются аппроксимационные и топологические методы
(см., например, [9], [10]). Краевая задача рассматривается как операторное уравнение. Используемые операторы часто не
обладают хорошими свойствами  , поэтому рассматривается некоторая аппроксимация этого уравнения. Затем разрешимость
этого аппроксимирующего уравнения исследуется в более сглаженном пространстве. Для этого применяется техника
топологической степени Лере-Шаудера. Последний шаг - предельный переход  в аппроксимирующем уравнении, поскольку
аппроксимирующие параметры стремятся к нулю, а решения аппроксимирующего уравнения сходятся к решению исходного
уравнения.


\section{Слабое условие}

Обозначим через $C_0^{\infty}(\Omega)^n$ пространство функций класса $C^{\infty}$, отображаемых $\Omega$ в $R^n$ с
компактным носителем в $\Omega$. Также нам потребуется определение следующих функциональных пространств
$$\mathcal{V}={v(x)=(v_1,...,v_n)\in C_0^{\infty}(\Omega)^n: div \ v = 0};$$
$V$ - замыкание на $\mathcal{V}$ по норме пространсва $W^1_2(\Omega)^n$
со скалырным произведением
$$((v,w))=\int\limits_{\Omega}\nabla v : \nabla w dx.$$
Здесь символ $\nabla v : \nabla w, v=(v_1,...,v_n), w=(w_1,...,w_n)$, обозначает покомпактное метрическое умножение
$$\nabla v : \nabla w = \sum_{i,j=1}^n \frac{\partial v_i}{\partial x_j}\frac{\partial w_i}{\partial x_j} $$
Пусть $X$ - замыкание $V$ относительно нормы пространства $W_2^3(\Omega)^n$. Рассмотрим пространство $X$ с нормой:
$$\parallel v\parallel_{X}=\bigg(\int\limits_{\Omega}\nabla(\Delta v):\nabla(\Delta v)dx\bigg)^{1/2}$$

\begin{definition}
    Пусть $f$ принадлежит $V^*$. Слабым решением краевой задачи (\ref{eq:1.6})"=(\ref{eq:1.8}) называется функция $v\in V$ такая,
    что для любого $y\in X$ она удовлетворяет равенству
    \begin{equation}\label{eq:2.1}
        \begin{gathered}
            \nu\int\limits_{\Omega}\nabla v: \nabla w dx-\int\limits_{\Omega} \sum_{i,j=1}^n v_iv_j\frac{\partial \varphi_j}
            {\partial x_i}dx-\varkappa\int\limits_{\Omega}\sum_{i,j,k=1}^n v_k\frac{\partial v_i}{\partial x_j}\frac{\partial^2 \varphi_i}{\partial x_j\partial x_k}dx- \\
            \varkappa\int\limits_{\Omega}\sum_{i,j,k=1}^n v_k\frac{\partial v_i}{\partial x_j}\frac{\partial^2 \varphi_i}
            {\partial x_j\partial x_k}dx+\\
            +2\varkappa\int\limits_{\Omega}(\mathcal{E}(v)W_p(v)-W_p(v)\mathcal{E}(v)):
            \nabla\varphi dx=\langle f,\varphi \rangle
        \end{gathered}
    \end{equation}
\end{definition}
Основным результатом статьи является следующая теорема.

\begin{theorem}
    Для любой $f\in V^*$ краевая задача (\ref{eq:1.6})"=(\ref{eq:1.8}) имеет хотя бы одно слабое решение $v_*\in V$
\end{theorem}

\section{Аппроксимационная задача}
При исследовании задачи (\ref{eq:1.6})"=(\ref{eq:1.8}) мы используем аппроксимационно"=топологический подход к задачам гидродинамики [10].
Фактически мы исследуем аппроксимирующую задачу с малым параметром $\varepsilon > 0$:

Аппроксимационная задача.

Найти функцию $v \in X$, которая для любого $\varphi \in X$ удовлетворяет следующему равенству
\begin{equation*}
    \begin{gathered}
        \varepsilon \int\limits_\Omega \nabla(\Delta v): \ \nabla(\Delta\varphi) \ dx -
        \int\limits_\Omega \sum_{i,j=1}^n v_i v_j \frac{\partial \varphi_j}{\partial x_i} dx \ \nu
        \int\limits_\Omega \nabla v: \ \nabla \varphi \ dx
    \end{gathered}
\end{equation*}

\begin{equation*}
    \begin{gathered}
        - \varkappa \int\limits_\Omega \sum_{i,j,k=1}^n v_k \frac{\partial v_i}{\partial x_j}
        \frac{\partial^2 \varphi_j}{\partial x_i \partial x_k} \ dx
        - \varkappa \int\limits_\Omega \sum_{i,j,k=1}^n v_k \frac{\partial v_j}{\partial x_i}
        \frac{\partial^2 \varphi_j}{\partial x_i \partial x_k} \ dx
    \end{gathered}
\end{equation*}

\begin{equation}\label{eq:3.1}
    \begin{gathered}
        + 2 \varkappa \int\limits_\Omega (\mathcal{E}(v) W_\rho (v) - W_\rho (v) \mathcal{E}(v)): \ \nabla \varphi
        = \langle f, \varphi \rangle.
    \end{gathered}
\end{equation}

Отметим, что (\ref{eq:3.1}) отличается от (\ref{eq:2.1}) наличием члена
\begin{equation*}
    \begin{gathered}
        \varepsilon \int\limits_\Omega \nabla(\Delta v): \ \nabla(\Delta\varphi) \ dx.
    \end{gathered}
\end{equation*}

На первом шаге мы получаем априорную оценку равенства (\ref{eq:3.1}) в пространстве $X$
и с помощью методов топологической степени показываем, что существует решение аппроксимирующей задачи в $X$.
Также получаем в пространстве $V$ оценку решения аппроксимирующей задачи, которая не зависит от параметра $\varepsilon$.
Затем простроим последовательность таких решений и покажем, что она допускает подпоследовательность,
сходящуюся к слабому решению краевой задачи (\ref{eq:1.6})"=(\ref{eq:1.8}) при стремлении параметра приближения $\varepsilon$ к нулю. 

Рассмотрим следующие операторы:

\begin{equation*}
    \begin{gathered}
        A: V \rightarrow V^*, \ \langle Av, \varphi \rangle = \int\limits_\Omega \nabla v: \ \nabla \varphi \ dx, \ v, \varphi \in V;
    \end{gathered}
\end{equation*}

\begin{equation*}
    \begin{gathered}
        N: X \rightarrow X^*, \ \langle Nv, \varphi \rangle = \int\limits_\Omega \nabla(\Delta v): \ \nabla(\Delta\varphi) \ dx, \ v, \varphi \in V;
    \end{gathered}
\end{equation*}

\begin{equation*}
    \begin{gathered}
        B_1: L_4(\Omega)^n \rightarrow V^*
        , \ \langle B_1(v), \varphi \rangle = \int\limits_\Omega \sum_{i,j=1}^n v_i v_j \frac{\partial \varphi_j}{\partial x_i} dx
        , \ v \in L_4(\Omega)^n, \ \varphi \in V;
    \end{gathered}
\end{equation*}

\begin{equation*}
    \begin{gathered}
        B_2: V \rightarrow X^*
        , \ \langle B_2(v), \varphi \rangle = \int\limits_\Omega \sum_{i,j,k=1}^n v_k \frac{\partial v_i}{\partial x_j}
        \frac{\partial^2 \varphi_j}{\partial x_i \partial x_k} \ dx
        , \ v \in V, \ \varphi \in X;
    \end{gathered}
\end{equation*}

\begin{equation*}
    \begin{gathered}
        B_3: V \rightarrow X^*
        , \ \langle B_3(v), \varphi \rangle = \int\limits_\Omega \sum_{i,j,k=1}^n v_k \frac{\partial v_j}{\partial x_i}
        \frac{\partial^2 \varphi_j}{\partial x_i \partial x_k} \ dx
        , \ v \in V, \ \varphi \in X;
    \end{gathered}
\end{equation*}

\begin{equation*}
    \begin{gathered}
        D: V \rightarrow X^*
        , \ \langle D(v), \varphi \rangle = \int\limits_\Omega (\mathcal{E}(v) W_\rho (v) - W_\rho (v) \mathcal{E}(v)): \ \nabla \varphi \ dx, \\
        v \in V, \ \varphi \in X;
    \end{gathered}
\end{equation*}

Поскольку в равенстве (\ref{eq:3.1}) функция $\varphi \in X$ произвольна, это соотношение эквивалентно следующему операторному уравнению:

\begin{equation}\label{eq:3.2}
    \begin{gathered}
        \varepsilon N v + \nu A v - B_1(v) - \varkappa B_2(v) - \varkappa B_3(v) + 2 \varkappa D(v) = f
    \end{gathered}
\end{equation}

Таким образом, слабым решением аппроксимирующей задачи является решение $\varphi \in X$ операторного уравнения (\ref{eq:3.2}).

Мы также определяем следующие операторы:

\begin{equation*}
    \begin{gathered}
        L: X \rightarrow X^*, \ L(v) = \varepsilon N v; \\
        K: X \rightarrow X^*, \ K(v) = \nu A v - B_1(v) - \varkappa B_2(v) - \varkappa B_3(v) + 2 \varkappa D(v).
    \end{gathered}
\end{equation*}

Задача нахождения решения уравнения (\ref{eq:3.2}) эквивалентна задаче нахождения решения для следующего операторного уравнения:
\begin{equation}\label{eq:3.3}
    \begin{gathered}
        L(v)+K(v)=f
    \end{gathered}
\end{equation}

Мы будем использовать следующие утверждения (доказательства леммы (\ref{lm:3.1}) - леммы (\ref{lm:3.4}) можно найти, например, в [5]).
\begin{lemma}\label{lm:3.1}
    Оператор $A:V\rightarrow V^*$ непрерывен и имеет место оценка 
    $$\parallel Av\parallel_{V^*}\leqslant C_1\parallel v\parallel_V.$$
    Более того, оператор $A:X\rightarrow X^*$ вполне непрерывен.
\end{lemma}

\begin{lemma}\label{lm:3.2}
    Оператор $L:X\rightarrow X^*$ непрерывен, обратим и для него имеет место оценка:
    $$\parallel Lv\parallel_{X^*}\leqslant\varepsilon\parallel v\parallel_v.$$
    Более того, оператор $L^{-1}:X^*\rightarrow X$ вполне непрерывен.
\end{lemma}

\begin{lemma}\label{lm:3.3}
    Для опреатора $B_1:L_4(\Omega)^n\varepsilon V^*$ непрерывен и имеет место следующая оценка:
    $$\parallel B_1v\parallel_{V^*}\leqslant C_2\parallel v\parallel^2_{L_4(\Omega)^n}.$$
    Более того, оператор $B_1:X\rightarrow X^*$ вполне непрерывен.
\end{lemma}

\begin{lemma}\label{lm:3.4}
    Отображение $B_i:V\rightarrow X^*, i=2,3$ непрерывно и имеет место следующая оценка:
    $$\parallel B_iv\parallel_{X^*}\leqslant C_3\parallel v\parallel^2_V.$$
    Боле того, оператор $B_i:X\rightarrow X^*$ полностью непрерывен.
\end{lemma}

\begin{lemma}\label{lm:3.5}
    Оператор $D:V\rightarrow X^*$ непрерывен и подчиняется оценке:
    \begin{equation}\label{eq:3.4}
        \begin{gathered}
            \parallel D(v)\parallel_{X^*} \leqslant C_4\parallel v\parallel^2_V.
        \end{gathered}
    \end{equation}
\end{lemma}

\begin{proof}
    Начнем о соценки $\mathcal{E}$ and $W_{\rho}$.
    $$\parallel\mathcal{E}\parallel^2_{L_2(\Omega)}=\sum_{i,j=1}^n \parallel\mathcal{E}_{i,j}(v)\parallel^2_{L_2(\Omega)}
    \leqslant С_5 \sum_{i,j=1}^n\int\limits_{\Omega}(\frac{\partial v_i}{\partial x_j}+\frac{\partial v_i}{\partial x_j})^2 dx$$
    $$=C_5\sum_{i,j=1}^n\int\limits_{\Omega}(\frac{\partial v_i}{\partial x_j}\frac{\partial v_i}{\partial x_j}+
    2\frac{\partial v_i}{\partial x_j}\frac{\partial v_j}{\partial x_i}+\frac{\partial v_j}{\partial x_j}\frac{\partial v_j}{\partial x_j})dx$$
    $$=C_5\sum_{i,j=1}^n\bigg[\int\limits_{\Omega}\frac{\partial v_i}{\partial x_j}\frac{\partial v_i}{\partial x_j}dx-
    2\int\limits_{\Omega} v_i\frac{\partial^2 v_j}{\partial x_i\partial x_j}dx+\int\limits_{\Omega}\frac{\partial v_j}{\partial x_i}\frac{\partial v_j}{\partial x_i}dx\bigg]$$
    $$=C_5\bigg[\int\limits_{\Omega}\nabla v : \nabla v dx +\int\limits_{\Omega}\nabla v : \nabla v dx\bigg]\leqslant
    2C_5\parallel v\parallel^2_V$$
    Далее, $\parallel\mathcal{E}\parallel_{L_2(\Omega)}\leqslant C_6\parallel v\parallel_V$.
    $$\parallel(W_{\rho})_{ij}(v)\parallel_{L_2(\Omega)}\leqslant\parallel(W_{\rho})_{ij}(v)\parallel_{L_{\infty}(\Omega)}$$
    $$\leqslant \frac{1}{2} \sup_{x\in\Omega}\bigg|\int\limits_{\Omega}\rho (x-y)(\frac{\partial v_i(y)}{\partial y_j}-\frac{\partial v_j(y)}{\partial y_i})dy\bigg|$$
    $$\leqslant \frac{1}{2}\sup\limits_{x\in\Omega} \bigg|-\frac{\partial\rho (x-y)}{\partial y_j}v_i(y)+\frac{\partial\rho (x-y)}{\partial y_i}v_i(y)dy\bigg|$$
    $$\leqslant\parallel grad\rho\parallel_{L_2(\Omega)}\parallel v(t)\parallel_{L_2(\Omega)}.$$
    По определению, для любого $v\in V, \varphi\in X$ имеем
    $$\mid\langle D(v),\varphi\rangle\mid=\bigg|\int\limits_{\Omega}\bigg(\mathcal{E}(v)W_{\rho}(v)-W_{\rho}(v)\mathcal{E}(v)\bigg):\nabla\varphi dx\bigg|\leqslant$$
    $$\leqslant C_7\bigg[\|\mathcal{E}(v)\|_{L_2(\Omega)}\|W_{\rho}(v)\|_{L_2(\Omega)}+\|W_{\rho}(v)\|_{L_2(\Omega)}\|\mathcal{E}(v)\|_{L_2(\Omega)}\bigg]\|\nabla\varphi\|_{C(\Omega)^n}\leqslant$$
    $$\leqslant C_8\| v\|^2_V\|\varphi\|_X$$
    Отсюда получаем оценку \ref{eq:3.3}.
    Теперь докажем, что оператор $D$ непрерывен. Для любых $v^m,v^0\in V$, имеем:
    $$\bigg|\langle D(v^m),\varphi\rangle-\langle D(v^0),\varphi\rangle\bigg|=\bigg|\int\limits_{\Omega}\bigg(\mathcal{E}(v^m)W_{\rho}(v^m)-W_{\rho}(v^m)\mathcal{E}(v^m)\bigg):\nabla\varphi$$
    $$-\int\limits_{\Omega}\bigg(\mathcal{E}(v^0)W_{\rho}(v^0)-W_{\rho}(v^0)\mathcal{E}(v^0)\bigg):\nabla\varphi dx\bigg|\leqslant C_9\bigg|\int\limits_{\Omega}\mathcal{E}(v^m)W_{\rho}(v^m)$$
    $$-W_{\rho}(v^m)\mathcal{E}(v^m)-\mathcal{E}(v^0)W_{\rho}(v^0)+W_{\rho}(v^0)\mathcal{E}(v^0)dx\bigg|\|\varphi\|_X$$
    $$\leqslant C_9\bigg|\int\limits_{\Omega}\mathcal{E}(v^m)\bigg(W_{\rho}(v^m)-W_{\rho}(v^0)\bigg)+\bigg(\mathcal{E}(v^m)-\mathcal{E}(v^0)\bigg)W_{\rho}(v^0)$$
    $$-W_{\rho}(v^m)\bigg(\mathcal{E}(v^m)-\mathcal{E}(v^0)\bigg)-\bigg(W_{\rho}(v^m)-W_{\rho}(v^0)\bigg)\mathcal{E}(v^0)dx\bigg|\|\varphi\|_X$$
    $$\leqslant C_{10}\bigg[\|\mathcal{E}(v^m)\|_{L_2(\Omega)}\|W_{\rho}(v^m-v^0)\|_{L_2(\Omega)}+\|\mathcal{E}(v^m-v^0)\|_{L_2(\Omega)}$$
    $$\times\|W_{\rho}(v^0)\|_{L_2(\Omega)}+\|W_{\rho}(v^m)\|_{L_2(\Omega)}\|\mathcal{E}(v^m-v^0)\|_{L_2(\Omega)}+\|W_{\rho}(v^m-v^0)\|_{L_2(\Omega)}$$
    $$\times\|\mathcal{E}(v^0)\|_{L_2(\Omega)}\bigg]\|\varphi\|_X\leqslant C_{11}\bigg[\|v^m\|_V\|v^m-v^0\|_V+\|v^m-v^0\|_V\|v^0\|_V$$
    $$+\|v^m\|_V\|v^m-v^0\|_V+\|v^m-v^0\|_V\|v^0\|_V\bigg]\|\varphi\|_X$$
    $$\leqslant C_{12}\bigg(\|v^m\|_V+\|v^0\|_V\bigg)\|v^m-v^0\|_V\|\varphi\|_X$$
    Таким образом, мы имеем $\|D(v^m)-D(v^0)\|_X\leqslant C_{13}(\|v^m\|_V+\|v^0\|_V)\|v^m-v^0\|_V$
    Пусть последовательность ${v^m}\subset V$ сходится к некторой функции $v^0\in V$. Тогда непрерывность отображения $D:V\rightarrow X$
    следует из предыдущего равенства.
\end{proof}

\begin{lemma}\label{lm:3.6}
    Оператор $K:X\rightarrow X^*$ вполне непрерывен.
\end{lemma}

\begin{proof}
    Полная непрерывность оператора $K:X\rightarrow X^*$ следует из полной непрерывности следующих операторов
    \begin{center}
        $A:X\rightarrow X^*$ \ Лемма \ref{lm:3.1} \\
        $B_1:X\rightarrow X^*$ \ Лемма \ref{lm:3.3} \\
        $B_2:X\rightarrow X^*$ \ Лемма \ref{lm:3.4} \\
        $B_3:X\rightarrow X^*$ \ Лемма \ref{lm:3.4} \\
        $D:X\rightarrow X^*$ \ Лемма \ref{lm:3.5} \\
    \end{center}
\end{proof}

На ряду с уравнением \ref{eq:3.3} рассмотрим следующее семейство операторных уравнений:
\begin{equation}\label{eq:3.5}
    \begin{gathered}
    L(v)+\lambda K(v)=\lambda f, \lambda\in [0,1]
    \end{gathered}
\end{equation}
что совпадает с уравнением \ref{eq:3.3} для $\lambda=1$.

\begin{theorem}\label{tm:3.1}
    Если $v\in X$ является решением операторного уравнения \ref{eq:3.5} для некоторых $\lambda\in [0,1]$, тогда справедлива следующая оценка:
    \begin{equation}\label{eq:3.6}
        \begin{gathered}
            \varepsilon\|v\|^2_X\leqslant C_{14}, \textrm{ где } C_{14}=\frac{\|f\|^2_{V^*}}{2\nu}
        \end{gathered}
    \end{equation}
    Более того, если $\lambda = 1$, тогда справедлива следующая оценка:
    \begin{equation}\label{eq:3.7}
        \begin{gathered}
            \nu\|v\|^2_X\leqslant C_{15}, \textrm{ где } C_{15}=\frac{\|f\|^2_{V^*}}{\nu}
        \end{gathered}
    \end{equation}
\end{theorem}

\begin{proof}
Пусть $X$ --- решение \ref{eq:3.5}. Тогда для любого $\varphi\in  X$ выполняется уравнение:
\begin{equation}\label{eq:3.8}
    \begin{gathered}
        \varepsilon\int\limits_{\Omega}\nabla (\Delta v):\nabla (\Delta\varphi)dx - \lambda\int\limits_{\Omega}\sum_{i,j=1}^n v_iv_j
        \frac{\partial\varphi_j}{\partial x_i}dx +\lambda\nu\int\limits_{\Omega}\nabla v : \nabla\varphi dx \\
        -\lambda\varkappa\int\limits_{\Omega}\sum_{i,j,k=1}^n v_k\frac{\partial v_i}{\partial x_j}\frac{\partial^2 \varphi_j}{\partial x_i\partial x_k}dx 
        -\lambda\varkappa\int\limits_{\Omega}\sum_{i,j,k=1}^n v_k\frac{\partial v_j}{\partial x_i}\frac{\partial^2 \varphi_j}{\partial x_i\partial x_k}dx \\
        +2\lambda\varkappa\int\limits_{\Omega}\bigg(\mathcal{E}(v)W_{\rho}(v)-W_{\rho}(v)\mathcal{E}\bigg):\nabla\varphi dx=\lambda\langle f, v\rangle
    \end{gathered}
\end{equation}
    Запишем, что  
    $$\int\limits_{\Omega}\sum_{i,j,k=1}^n v_k\frac{\partial v_i}{\partial x_j}\frac{\partial^2 \varphi_j}{\partial x_i\partial x_k}dx+
    \int\limits_{\Omega}\sum_{i,j,k=1}^n v_k\frac{\partial v_j}{\partial x_i}\frac{\partial^2 \varphi_j}{\partial x_i\partial x_k}dx $$
    $$= 2\int\limits_{\Omega}\sum_{i,j,k=1}^n v_k(t)\mathcal{E}_{ij}(v)\frac{\partial^2 \varphi_j}{\partial x_i\partial x_k}dx=-2
    \int\limits_{\Omega}\sum_{i,j,k=1}^n v_k\frac{\partial\mathcal{E}_{ij}(v)}{\partial x_k}\frac{\partial\varphi_j}{\partial x_i}dx $$
    $$-2\int\limits_{\Omega}\sum_{i,j,k=1}^n\frac{\partial v_k}{\partial x_k}\mathcal{E}_{ij}(v)\frac{\partial\varphi_j}{\partial x_i}dx
    =-2\int\limits_{\Omega}\sum_{i,j,k=1}^n v_k \frac{\partial\mathcal{E}_{ij}(v)}{\partial x_k}\frac{\partial\varphi_j}{\partial x_i}dx.$$
    Следовательно, \ref{eq:3.8} запишется в форме
    $$\varepsilon\int\limits_{\Omega}\nabla(\Delta v):\nabla(\Delta\varphi)dx-\lambda\int\limits_{\Omega}\sum_{i,j,k=1}^n v_iv_j\frac{\partial\varphi_j}{\partial x_i}dx$$
    $$+\lambda\nu\int\limits_{\Omega}\nabla v:\nabla\varphi dx+2\lambda\nu\int\limits_{\Omega}\sum_{i,j,k=1}^n v_k\frac{\partial\mathcal{E}_{ij}(v)}{\partial x_k}\frac{\partial\varphi_j}{\partial x_i}dx$$
    $$+2\lambda\varkappa\int\limits_{\Omega}\bigg(\mathcal{E}(v)W_{\rho}(v)-W_{\rho}(v)\mathcal{E}(v)\bigg):\nabla\varphi dx=\lambda\langle f, v\rangle$$
    Поскольку последнее равенство выполняется для всех $\varphi\in X$, это верно для $\varphi=v$ так же:
    \begin{equation}\label{eq:3.9}
        \begin{gathered}
            \varepsilon\int\limits_{\Omega}\nabla(\Delta v):\nabla(\Delta\varphi)dx-\lambda\int\limits_{\Omega}\sum_{i,j,k=1}^n v_iv_j\frac{\partial v_j}{\partial x_i}dx \\
            +\lambda\nu\int\limits_{\Omega}\nabla v:\nabla\varphi+2\lambda\varkappa\int\limits_{\Omega}\sum_{i,j,k=1}^n v_k\frac{\partial\mathcal{E}_{ij}(v)}{\partial x_k}\frac{\partial\varphi_j}{\partial x_i}dx \\
            +2\lambda\varkappa\int\limits_{\Omega}\bigg(\mathcal{E}(v)W_{\rho}(v)-W_{\rho}(v)\mathcal{E}(v)\bigg):\nabla v dx=\lambda\langle f, v\rangle.
        \end{gathered}
    \end{equation}
Сведем слогаемыемв левой части уравнения \ref{eq:3.9} следующим образом:
$$\varepsilon\int\limits_{\Omega}\nabla(\Delta v):\nabla(\Delta\varphi)dx=\varepsilon\|v\|^2_X;$$
$$\int\limits_{\Omega}\sum_{i,j=1}^n v_iv_j\frac{\partial v_j}{\partial x_i}dx=\int\limits_{\Omega}\sum_{i,j=1}^n v_i
\frac{\partial (v_jv_j)}{\partial x_i}dx=-\int\limits_{\Omega}\sum_{i,j=1}^n\frac{\partial v_i}{\partial x_i}v_jv_jdx=0$$
$$\int\limits_{\Omega}\bigg(\mathcal{E}(v)W_{\rho}(v)-W_{\rho}(v)\mathcal{E}(v)\bigg):\nabla v dx=\frac{1}{2}\int\limits_{\Omega}\bigg(\mathcal{E}(v)
W_{\rho}(v)-W_{\rho}(v)\mathcal{E}(v)\bigg):$$
$$:\bigg(\mathcal{E}(v)+W(v)\bigg)dx=\frac{1}{2}\int\limits_{\Omega}\bigg(\mathcal{E}(v)W_{\rho}(v)-W_{\rho}(v)\mathcal{E}(v)\bigg):\mathcal{E}(v)dx$$
$$+\frac{1}{2}\int\limits_{\Omega}\bigg(\mathcal{E}(v)W_{\rho}(v)-W_{\rho}(v)\mathcal{E}(v)\bigg):W(v)dx=\frac{1}{2}\sum_{i,j,k=1}^n\int\limits_{\Omega}
\mathcal{E}_{ij}(W_{\rho})_{jk}\mathcal{E}_{ik}$$
$$-(W_{\rho})_{jk}\mathcal{E}_{ki}\mathcal{E}_{ji}\bigg)dx+\frac{1}{2}\sum_{i,j,k=1}^n\int\limits_{\Omega}\bigg(\mathcal{E}_{ij}(W_{\rho})_{jk}W_{ik}-
(W_{\rho})_{kj}\mathcal{E}_{ji}W_{ki}\bigg) dx$$
$$\frac{1}{2}\sum_{i,j,k=1}^n\int\limits_{\Omega}\mathcal{E}_{ij}(W_{\rho})_{jk}\mathcal{E}_{ik}-\mathcal{E}_{ij}(W_{\rho})_{jk}\mathcal{E}_{ik}+
\frac{1}{2}\sum_{i,j,k=1}^n\int\limits_{\Omega}\mathcal{E}_{ij}(W_{\rho})_{jk}W_{ik}$$
$$-\mathcal{E}_{ij}(W_{\rho})_{jk}W_{ik}dx=0;$$
$$\int\limits_{\Omega}\sum_{i,j,k=1}^n v_k\frac{\partial\mathcal{E}_{ij}(v)}{\partial x_k}\frac{\partial v_j}{\partial x_i}dx=\frac{1}{2}\bigg(
\int\limits_{\Omega}\sum_{i,j,k=1}^n v_k\frac{\partial\mathcal{E}_{ij}(v)}{\partial x_k}\frac{\partial v_j}{\partial x_i}dx$$
$$+\int\limits_{\Omega}\sum_{i,j,k=1}^nv_k\frac{\partial\mathcal{E}_{ij}(v)}{\partial x_k}\frac{\partial v_i}{\partial x_j}dx\bigg)=
\int\limits_{\Omega}\sum_{i,j,k=1}^n v_k\frac{\partial (\mathcal{E}_{ij}(v)\mathcal{E}_{ij}(v))}{\partial x_k}dx$$
$$=-\int\limits_{\Omega}\sum_{k=1}^n \frac{\partial v_k}{\partial x_k}\sum_{i,j=1}^n\mathcal{E}_{ij}(v)\mathcal{E}_{ij}(v)dx=0.$$
Здесь указано, что тензор скорости дифформации $\mathcal{E}(u)$ симметричен, а тензоры $W_{\rho}(v)$ и $W(v)$ кососимметричны. 
Следовательно, уравнения \ref{eq:3.9} переписать в следующем виде:
$$\varepsilon\|v\|^2_X+\lambda\nu\|v\|^2_V=\lambda\langle f,v\rangle$$
Используя верхнюю оценку правой части последнего уравнения.
$$\lambda\langle f,v\rangle\leqslant\lambda |\langle f,v\rangle|\leqslant\lambda\|f\|_{V^*}\|v\|_V\leqslant\|f\|_{V^*}\|v\|_V\leqslant
\lambda\frac{\|f\|_{V^*}^2}{2\delta}+\lambda\frac{\delta\|v\|_V^2}{2}$$
для $\delta=\nu$ имеем
$$\varepsilon\|v\|^2_X+\lambda\nu\|v\|^2_V\leqslant\lambda\frac{\|f\|^2_{V^*}}{2\nu}+\lambda\frac{\nu\|v\|^2_V}{2},$$
$$\varepsilon\|v\|^2_X+\lambda\frac{\nu\|v\|^2_V}{2}\leqslant\lambda\frac{\|f\|^2_{V^*}}{2\nu}, \varepsilon\|v\|^2_X\leqslant\lambda
\frac{\|f\|^2_{V^*}}{2\nu}\leqslant\frac{\|f\|^2_{V^*}}{2\nu}$$
Аналогично для $\lambda=1$ имеем $\nu\|v\|^2_V\leqslant\lambda\frac{\|f\|^2_{V^*}}{\nu}$. Это доказывает \ref{eq:3.6} и \ref{eq:3.7}.
\end{proof}

\begin{theorem}
    Операторное уравнение \ref{eq:3.3} имеет хотя бы одно слабое решение $v\in X:$
\end{theorem}

\begin{proof}
    Для доказательства этой теоремы мы используем теорию степени Лере-Шаудера для вполне непрерывных 
    векторных полей. В силу априорной оценки \ref{eq:3.6}, все решения семейства уравнений \ref{eq:3.5} содержатся в шаре 
    $B_R \subset X$ радиуса $R = C_{14} +1$. По лемме \ref{lm:3.6} отображение $[-K (\cdot) + f]: X \rightarrow X^*$
    вполне непрерывно. В силу леммы\ref{lm:3.2} оператор $L^{-1}: X^*\rightarrow X$ непрерывен.

    Таким образом, отображение $L^{-1}[-K (\cdot) + f]: X \rightarrow X$ вполне непрерывно.  
    Тогда отображение $G:[0, 1] \times X\rightarrow X, G (\lambda, v) = \lambda L^{-1}[- K (v) + f]$ 
    полностью непрерывно по двумерному аргументу $(\lambda, v)$.  Из сказанного выше получаем, что вполне непрерывное 
    векторное поле $\Phi(\lambda, v) = v - G (\lambda, v)$ не обращается в нуль на границе $B_R$. По гомотопической 
    инвариантности степени получаем
    $$deg_{LS}(\Phi(0,\cdot),B_R,0)=deg_{LS}(\Phi(1,\cdot),B_R,0).$$
    Напомним, что $\Phi(0,\cdot) = I$ и по свойству нормализации степени $deg_{LS} (I, B_R, 0) = 1$. 
    Следовательно, $deg_{LS} ((1,\cdot), BR.0) = 1$. 
    
    Таким образом, мы видим, что существует по крайней мере решение $v\in X$ уравнение
    $$v-L^{-1}[-K(v)+f]=0$$
    и следовательно уравнение \ref{eq:3.3}.

    Поскольку существует решение $v\in X$ уравнения \ref{eq:3.3}, из сказанного выше следует, 
    что аппроксимирующая задача имеет хотя бы одно слабое решение $v\in X$.
\end{proof}

\section{Доказательство теоремы 2.1}

\begin{proof}
    В (\ref{eq:2.1}) возьмем $\varepsilon_m = \frac{1}{m}$. Последовательность $\{\varepsilon_m\}$ сходится к нулю при $m \rightarrow +\infty$.
    По теореме 3.2 для любого $\varepsilon_m$ существует слабое решение $v_m \in X \subset V$ задачи аппроксимации.
    Таким образом, каждое $v_m$ удовлетворяет уравнению

    \begin{equation}\label{eq:4.1}
        \begin{gathered}
            \varepsilon_m \int\limits_{\Omega} \nabla (\Delta v_m): \nabla (\Delta \varphi) dx -
            \int\limits_{\Omega}\sum_{i,j=1}^n (v_m)_i (v_m) \frac{\partial \varphi_j}{\partial x_i} dx + \nu
            \int\limits_{\Omega} \nabla v_m: \nabla \varphi dx\\
            -\varkappa \int\limits_{\Omega}\sum_{i,j,k=1}^n (v_m)_k
            \frac{\partial (v_m)_i}{\partial x_j} \frac{\partial^2 \varphi_j}{\partial x_i \partial x_k} dx - \varkappa \int\limits_{\Omega}\sum_{i,j,k=1}^n
            (v_m)_k \frac{\partial (v_m)_j}{\partial x_i} \frac{\partial^2 \varphi_j}{\partial x_i \partial x_k} dx\\
            +2\varkappa \int\limits_{\Omega} \bigg(\mathcal{E}(v_m)W_\rho (v_m) - W_\rho(v_m)\mathcal{E}(v_m)\bigg): \nabla dx =
            \langle f, \varphi \rangle.
        \end{gathered}
    \end{equation}

    Тогда без ограничения общности (переходя при необходимости к последовательности) из (3.6) получаем, что
    $$\lim_{m \rightarrow \infty}\bigg| \varepsilon_m \int\limits_{\Omega} \nabla (\Delta v_m): \nabla (\Delta \varphi) dx \bigg| =$$
    $$=\lim_{m \rightarrow \infty} \sqrt{\varepsilon_m} \lim_{m \rightarrow \infty}\bigg|\sqrt{\varepsilon_m}
    \int\limits_{\Omega} \nabla (\Delta v_m): \nabla (\Delta \varphi) dx \bigg|$$
    так что мы получаем
    $$\varepsilon_m \int\limits_{\Omega} \nabla (\Delta v_m): \nabla (\Delta \varphi) dx \rightarrow 0, \ m \rightarrow +\infty$$

    Для остальных интегралов имеем
    \begin{equation*}
        \begin{gathered}
            \varkappa \int\limits_{\Omega}\sum_{i,j,k=1}^n (v_m)_k
            \frac{\partial (v_m)_i}{\partial x_j} \frac{\partial^2 \varphi_j}{\partial x_i \partial x_k} dx \rightarrow
            \varkappa \int\limits_{\Omega}\sum_{i,j,k=1}^n (v_*)_k
            \frac{\partial (v_*)_i}{\partial x_j} \frac{\partial^2 \varphi_j}{\partial x_i \partial x_k} dx,\\
            m \rightarrow +\infty;
        \end{gathered}
    \end{equation*}
    \begin{equation*}
        \begin{gathered}
            \varkappa \int\limits_{\Omega}\sum_{i,j,k=1}^n (v_m)_k
            \frac{\partial (v_m)_j}{\partial x_i} \frac{\partial^2 \varphi_j}{\partial x_i \partial x_k} dx \rightarrow
            \varkappa \int\limits_{\Omega}\sum_{i,j,k=1}^n (v_*)_k
            \frac{\partial (v_*)_j}{\partial x_i} \frac{\partial^2 \varphi_j}{\partial x_i \partial x_k} dx,\\
            m \rightarrow +\infty;
        \end{gathered}
    \end{equation*}
    Действительно, здесь последовательность $v_m$ сходится к $v_*$ cильно в $L_4(\Omega)^n$,
    а $\nabla(v_m)$ сходится к $\nabla v_*$ слабо в $L_2(\Omega)^{n^2}$.
    Таким образом, результат сходится к произведению пределов. В итоге мы имеем
    \begin{equation*}
        \begin{gathered}
            \int\limits_{\Omega} \bigg(\mathcal{E}(v_m)W_\rho (v_m) - W_\rho(v_*)\mathcal{E}(v_*)\bigg): \nabla\varphi dx\\
            = \int\limits_{\Omega} \bigg(\mathcal{E}(v_m)(W_\rho (v_m) - W_\rho(v_*))
            + (\mathcal{E}(v_m) - \mathcal{E}(v_*))W_\rho (v_*)\bigg): \nabla\varphi dx\\
            \leq ||\mathcal{E}(v_m)||_{L_2(\Omega)}||\nabla\varphi||_{L_2(\Omega)}||W_\rho(v_m - v_*)||_{L_\infty(\Omega)} +\\
            + ||W_\rho(v_*)||_{L_\infty(\Omega)} \int\limits_{\Omega}\mathcal{E}(v_m - v_m): \nabla\varphi dx\\
            \leq ||\mathcal{E}(v_m)||_{L_2(\Omega)}||\nabla\varphi||_{L_2(\Omega)}||(v_m - v_*)||_{L_2(\Omega)} +\\
            + ||W_\rho(v_*)||_{L_\infty(\Omega)} \int\limits_{\Omega}\mathcal{E}(v_m - v_m): \nabla\varphi dx\\
            \leq ||\mathcal{E}(v_m)||_{L_2(\Omega)}||\nabla\varphi||_{L_2(\Omega)}||(v_m - v_*)||_{L_4(\Omega)} +\\
            + ||W_\rho(v_*)||_{L_\infty(\Omega)} \int\limits_{\Omega}\mathcal{E}(v_m - v_m): \nabla\varphi dx.
        \end{gathered}
    \end{equation*}
    Напомним, что последовательность $v_m$ сходится к $v_*$ сильно в $L_4(\Omega)^n$,
    а $\nabla(v_m)$ сходится к $\nabla v_*$ слабо в $L_2(\Omega)^{n^2}$.
    Следовательно, мы имеем
    $$\int\limits_{\Omega}\mathcal{E}(v_m) W_\rho(v_m): \nabla\varphi dx \rightarrow
    \int\limits_{\Omega}\mathcal{E}(v_*) W_\rho(v_*): \nabla\varphi dx, \ m \rightarrow +\infty$$

    Аналогично получаем
    $$\int\limits_{\Omega}W_\rho(v_m)\mathcal{E}(v_m): \nabla\varphi dx \rightarrow
    \int\limits_{\Omega}W_\rho(v_*)\mathcal{E}(v_*): \nabla\varphi dx, \ m \rightarrow +\infty$$

    Таким образом, переходя к пределу в уравнении (\ref{eq:4.1}) при $m \rightarrow +\infty$,
    мы видим, что предельная функция $v_*$ удовлетворяет следующему уравнению:
    \begin{equation*}
        \begin{gathered}
            \nu\int\limits_{\Omega}\nabla v_*: \nabla\varphi dx -
            \int\limits_{\Omega}\sum_{i,j=1}^n (v_*)_i (v_*)_j \frac{\partial \varphi_j}{\partial x_i} dx
            - \varkappa \int\limits_{\Omega}\sum_{i,j,k=1}^n (v_*)_k \frac{\partial (v_*)_i}{\partial x_j}
            \frac{\partial^2 \varphi_j}{\partial x_i \partial x_k} dx
        \end{gathered}
    \end{equation*}
    \begin{equation*}
        \begin{gathered}
            - \varkappa \int\limits_{\Omega}\sum_{i,j,k=1}^n (v_*)_k \frac{\partial (v_*)_i}{\partial x_j}
            \frac{\partial^2 \varphi_j}{\partial x_i \partial x_k} dx + \\ + 2 \varkappa
            \int\limits_{\Omega}\bigg(\mathcal{E}(v_*) W_\rho(v_*) - W_\rho(v_*)\mathcal{E}(v_*)\bigg): \nabla\varphi dx
        \end{gathered}
    \end{equation*}
    Это доказывает, что $v_* \in V$. Это завершает доказательство теоремы 2.1.
\end{proof}
