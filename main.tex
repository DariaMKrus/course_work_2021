\begin{equation}\label{eq:1.6}
    \begin{gathered}
        \sum_{i=1}^n v_i\frac{\partial v}{\partial x_j}-\nu\Delta v- 2\varkappa Div\bigg(v_k\frac{\partial\mathcal{E}(v)}
        {\partial x_k}\bigg)-\\-2\varkappa Div\bigg(\mathcal{E}(v)W_p(v)-W_p(v)\mathcal{E}(v)\bigg)+grad p=f, x\in\Omega
    \end{gathered}
\end{equation}

\begin{equation}\label{eq:1.7}
    \begin{gathered}
        div \ v=0, x\in\Omega
    \end{gathered}
\end{equation}

Для системы (\ref{eq:1.6})"=(\ref{eq:1.7}) рассмотрим краетвую задачу с граничным условием
\begin{equation}\label{eq:1.8}
     \begin{gathered}
         v\mid_{\partial\Omega}=0.
     \end{gathered}
\end{equation}

В настоящей работе исследуется существование слабого решения краевой задачи (\ref{eq:1.6})"=(\ref{eq:1.8}), описывающего движение слабых
водных растворов полимеров, заполняющих ограниченную область $\Omega \in R^n$, $n = 2,3$, которое определяется реологическим
соотношением  со сглаженной производной Яуманна.

Для исследования используются аппроксимационные и топологические методы
(см., например, [9], [10]). Краевая задача рассматривается как операторное уравнение. Используемые операторы часто не
обладают хорошими свойствами  , поэтому рассматривается некоторая аппроксимация этого уравнения. Затем разрешимость
этого аппроксимирующего уравнения исследуется в более сглаженном пространстве. Для этого применяется техника
топологической степени Лере-Шаудера. Последний шаг - предельный переход  в аппроксимирующем уравнении, поскольку
аппроксимирующие параметры стремятся к нулю, а решения аппроксимирующего уравнения сходятся к решению исходного
уравнения.


\section{Слабое условие}

Обозначим через $C_0^{\infty}(\Omega)^n$ пространство функций класса $C^{\infty}$, отображаемых $\Omega$ в $R^n$ с
компактным носителем в $\Omega$. Также нам потребуется определение следующих функциональных пространств
$$\mathcal{V}={v(x)=(v_1,...,v_n)\in C_0^{\infty}(\Omega)^n: div \ v = 0};$$
$V$ - замыкание на $\mathcal{V}$ по норме пространсва $W^1_2(\Omega)^n$
со скалырным произведением
$$((v,w))=\int\limits_{\Omega}\nabla v : \nabla w dx.$$
Здесь символ $\nabla v : \nabla w, v=(v_1,...,v_n), w=(w_1,...,w_n)$, обозначает покомпактное метрическое умножение
$$\nabla v : \nabla w = \sum_{i,j=1}^n \frac{\partial v_i}{\partial x_j}\frac{\partial w_i}{\partial x_j} $$
Пусть $X$ - замыкание $V$ относительно нормы пространства $W_2^3(\Omega)^n$. Рассмотрим пространство $X$ с нормой:
$$\parallel v\parallel_{X}=\bigg(\int\limits_{\Omega}\nabla(\Delta v):\nabla(\Delta v)dx\bigg)^{1/2}$$

\begin{definition}
    Пусть $f$ принадлежит $V^*$. Слабым решением краевой задачи (\ref{eq:1.6})"=(\ref{eq:1.8}) называется функция $v\in V$ такая,
    что для любого $y\in X$ она удовлетворяет равенству
    \begin{equation}\label{eq:2.1}
        \begin{gathered}
            \nu\int\limits_{\Omega}\nabla v: \nabla w dx-\int\limits_{\Omega} \sum_{i,j=1}^n v_iv_j\frac{\partial \varphi_j}
            {\partial x_i}dx-\varkappa\int\limits_{\Omega}\sum_{i,j,k=1}^n v_k\frac{\partial v_i}{\partial x_j}\frac{\partial^2 \varphi_i}{\partial x_j\partial x_k}dx- \\
            \varkappa\int\limits_{\Omega}\sum_{i,j,k=1}^n v_k\frac{\partial v_i}{\partial x_j}\frac{\partial^2 \varphi_i}
            {\partial x_j\partial x_k}dx+ 2\varkappa\int\limits_{\Omega}(\mathcal{E}(v)W_p(v)-W_p(v)\mathcal{E}(v)):
            \nabla\varphi dx=\langle f,\varphi \rangle
        \end{gathered}
    \end{equation}
\end{definition}
Основным результатом статьи является следующая теорема.

\begin{theorem}
    Для любой $f\in V^*$ краевая задача (\ref{eq:1.6})"=(\ref{eq:1.8}) имеет хотя бы одно слабое решение $v_*\in V$
\end{theorem}

\section{Аппроксимационная задача}
При исследовании задачи (\ref{eq:1.6})"=(\ref{eq:1.8}) мы используем аппроксимационно"=топологический подход к задачам гидродинамики [10].
Фактически мы исследуем аппроксимирующую задачу с малым параметром $\varepsilon > 0$:

Аппроксимационная задача.

Найти функцию $v \in X$, которая для любого $\varphi \in X$ удовлетворяет следующему равенству
\begin{equation*}
    \begin{gathered}
        \varepsilon \int\limits_\Omega \nabla(\Delta v): \ \nabla(\Delta\varphi) \ dx -
        \int\limits_\Omega \sum_{i,j=1}^n v_i v_j \frac{\partial \varphi_j}{\partial x_i} dx \ \nu
        \int\limits_\Omega \nabla v: \ \nabla \varphi \ dx
    \end{gathered}
\end{equation*}

\begin{equation*}
    \begin{gathered}
        - \varkappa \int\limits_\Omega \sum_{i,j,k=1}^n v_k \frac{\partial v_i}{\partial x_j}
        \frac{\partial^2 \varphi_j}{\partial x_i \partial x_k} \ dx
        - \varkappa \int\limits_\Omega \sum_{i,j,k=1}^n v_k \frac{\partial v_j}{\partial x_i}
        \frac{\partial^2 \varphi_j}{\partial x_i \partial x_k} \ dx
    \end{gathered}
\end{equation*}

\begin{equation}\label{eq:3.1}
    \begin{gathered}
        + 2 \varkappa \int\limits_\Omega (\mathcal{E}(v) W_\rho (v) - W_\rho (v) \mathcal{E}(v)): \ \nabla \varphi
        = \langle f, \varphi \rangle.
    \end{gathered}
\end{equation}

Отметим, что (\ref{eq:3.1}) отличается от (\ref{eq:2.1}) наличием члена
\begin{equation*}
    \begin{gathered}
        \varepsilon \int\limits_\Omega \nabla(\Delta v): \ \nabla(\Delta\varphi) \ dx.
    \end{gathered}
\end{equation*}

На первом шаге мы получаем априорную оценку равенства (\ref{eq:3.1}) в пространстве $X$
и с помощью методов топологической степени показываем, что существует решение аппроксимирующей задачи в $X$.
Также получаем в пространстве $V$ оценку решения аппроксимирующей задачи, которая не зависит от параметра $\varepsilon$.
Затем простроим последовательность таких решений и покажем, что она допускает подпоследовательность,
сходящуюся к слабому решению краевой задачи (\ref{eq:1.6})"=(\ref{eq:1.8}) при стремлении параметра приближения $\varepsilon$ к нулю. 

Рассмотрим следующие операторы:

\begin{equation*}
    \begin{gathered}
        A: V \rightarrow V^*, \ \langle Av, \varphi \rangle = \int\limits_\Omega \nabla v: \ \nabla \varphi \ dx, \ v, \varphi \in V;
    \end{gathered}
\end{equation*}

\begin{equation*}
    \begin{gathered}
        N: X \rightarrow X^*, \ \langle Nv, \varphi \rangle = \int\limits_\Omega \nabla(\Delta v): \ \nabla(\Delta\varphi) \ dx, \ v, \varphi \in V;
    \end{gathered}
\end{equation*}

\begin{equation*}
    \begin{gathered}
        B_1: L_4(\Omega)^n \rightarrow V^*
        , \ \langle B_1(v), \varphi \rangle = \int\limits_\Omega \sum_{i,j=1}^n v_i v_j \frac{\partial \varphi_j}{\partial x_i} dx
        , \ v \in L_4(\Omega)^n, \ \varphi \in V;
    \end{gathered}
\end{equation*}

\begin{equation*}
    \begin{gathered}
        B_2: V \rightarrow X^*
        , \ \langle B_2(v), \varphi \rangle = \int\limits_\Omega \sum_{i,j,k=1}^n v_k \frac{\partial v_i}{\partial x_j}
        \frac{\partial^2 \varphi_j}{\partial x_i \partial x_k} \ dx
        , \ v \in V, \ \varphi \in X;
    \end{gathered}
\end{equation*}

\begin{equation*}
    \begin{gathered}
        B_3: V \rightarrow X^*
        , \ \langle B_3(v), \varphi \rangle = \int\limits_\Omega \sum_{i,j,k=1}^n v_k \frac{\partial v_j}{\partial x_i}
        \frac{\partial^2 \varphi_j}{\partial x_i \partial x_k} \ dx
        , \ v \in V, \ \varphi \in X;
    \end{gathered}
\end{equation*}

\begin{equation*}
    \begin{gathered}
        D: V \rightarrow X^*
        , \ \langle D(v), \varphi \rangle = \int\limits_\Omega (\mathcal{E}(v) W_\rho (v) - W_\rho (v) \mathcal{E}(v)): \ \nabla \varphi \ dx, \\
        v \in V, \ \varphi \in X;
    \end{gathered}
\end{equation*}

Поскольку в равенстве (\ref{eq:3.1}) функция $\varphi \in X$ произвольна, это соотношение эквивалентно следующему операторному уравнению:

\begin{equation}\label{eq:3.2}
    \begin{gathered}
        \varepsilon N v + \nu A v - B_1(v) - \varkappa B_2(v) - \varkappa B_3(v) + 2 \varkappa D(v) = f
    \end{gathered}
\end{equation}

Таким образом, слабым решением аппроксимирующей задачи является решение $\varphi \in X$ операторного уравнения (\ref{eq:3.2}).

Мы также определяем следующие операторы:

\begin{equation*}
    \begin{gathered}
        L: X \rightarrow X^*, \ L(v) = \varepsilon N v; \\
        K: X \rightarrow X^*, \ K(v) = \nu A v - B_1(v) - \varkappa B_2(v) - \varkappa B_3(v) + 2 \varkappa D(v).
    \end{gathered}
\end{equation*}

Задача нахождения решения уравнения (\ref{eq:3.2}) эквивалентна задаче нахождения решения для следующего операторного уравнения:
\begin{equation}\label{eq:3.3}
    \begin{gathered}
        L(v)+K(v)=f
    \end{gathered}
\end{equation}

Мы будем использовать следующие утверждения (доказательства леммы (\ref{lm:3.1}) - леммы (\ref{lm:3.4}) можно найти, например, в [5]).
\begin{lemma}\label{lm:3.1}
    Оператор $A:V\rightarrow V^*$ непрерывен и имеет место оценка 
    $$\parallel Av\parallel_{V^*}\leqslant C_1\parallel v\parallel_V.$$
    Более того, оператор $A:X\rightarrow X^*$ вполне непрерывеню
\end{lemma}

\begin{lemma}\label{lm:3.2}
    Оператор $L:X\rightarrow X^*$ непрерывен, обратим и для него имеет место оценка:
    $$\parallel Lv\parallel_{X^*}\leqslant\varepsilon\parallel v\parallel_v.$$
\end{lemma}

\begin{lemma}\label{lm:3.3}
    Для опреатора $B_1:L_4(\Omega)^n\varepsilon V^*$ непрерывен и имеет место следующая оценка:
    $$\parallel B_1v\parallel_{V^*}\leqslant C_2\parallel v\parallel^2_{L_4(\Omega)^n}.$$
    Более того, оператор $B_1:X\rightarrow X^*$ вполне непрерывен.
\end{lemma}

\begin{lemma}\label{lm:3.4}
    Отображение $B_i:V\rightarrow X^*, i=2,3$ непрерывно и имеет место следующая оценка:
    $$\parallel B_iv\parallel_{X^*}\leqslant C_3\parallel v\parallel^2_V.$$
    Боле того, оператор $B_i:X\rightarrow X^*$ полностью непрерывен.
\end{lemma}